% this file is called up by main.tex
% content in this file will be fed into the main document

% ---------------------------------------------------------------------------

A cryptocurrency blockchain is commonly understood as a public distributed ledger containing transactions 
verifiable by third parties, be it the mining community or the public in general.
It would seem that transactions would need to be sent and stored in clear text format in order to
make them publicly verifiable.
\\

As we will show in this thesis, this is an incorrect assumption. It is indeed possible to use cryptographic
artifacts to conceal participants of transactions as well as the amounts involved. And yet, allow transactions
to be verified and consensuated by the mining community.
\\

Furthermore, we will also show that transaction privacy does not automatically entail lawlessness nor
a total lack of insight. There are mechanisms built into the cryptocurrency studied here that allow for selective
access to transactions by, say, authorities, without resulting in a conflict with user privacy.




