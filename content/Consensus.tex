

\chapter{Consensus in the blockchain}
\label{chapter:consensus}

A blockchain is a public distributed database, where every peer can freely submit blocks.
Peers do not know each other, and do not have to trust each other.
Nevertheless, they must reach a common understanding, or consensus, as to what constitutes
the truth in the blockchain.
After all, the latter contains transactions with valuables, so reaching a consensus is crucial.

The most widely known and successful consensus algorithms among cryptocurrencies apply
the {\em Proof-of-Work} principle.

Briefly explained, it means that peers ({\em miners}) have to invest computing effort
--- and prove they have done it ---
in order to submit a new block while standing a chance that it is accepted by peers.
Another peer wanting to submit a different block would need to invest that same amount of
computing power before submission.
If this peer would want to submit an alternative chain of blocks, then the computing 
effort needed would increase further.

The philosophy behind the proof-of-work principle is that being dishonest should not pay off,
as a dishonest node would essentially need to have at least the computing power of the rest of the network
in order to compete with it.


\section{CryptoNight}

Monero uses also a proof-of-work algorithm for consensus. This is a proven approach that
is used in the most successful cryptocurrencies.
However, the particular algorithm, CryptoNight, has design traits that distinguish it from 
other currencies.


\subsection{The Bitcoin proof-of-work}

Bitcoin requires that miners find a block value (called {\em nonce}) such that the double SHA256
hash of the block be less than a certain pre-defined target. Give the apparent unpredictability
of the SHA256 hashing algorithm, the target will control the hashing power necessary
to submit a new block.

The SHA256 hash function is very CPU bound, which in turn means that it is possible to devise
ASICs (Application-Specific Integrated Circuit) for the specific task.

Currently, mining on commodity hardware is no longer viable, and all the mining power is
in the hands of rather few actors, who can invest in large ASIC farms. Even worse, many such
actors work in mining pools, leading to even further concentration.

This concentration of mining power in the hands of few actors results in greater
vulnerability to attacks requiring control of large amounts of computing power.
In open blockchains, mining power should be as distributed as possible. After all,
reaching consensus in the blockchain can be compared to voting.


\subsection{Memory-bound over CPU bound}

The CryptoNight algorithm aims at being more egalitarian, so that commodity hardware can be used
for mining. Mining power should grow as linearly as possible respect to investments in hardware.
This would facilitate a wider and more varied mining community, which in turn should yield a more secure
cryptocurrency.

A full description of the algorithm can be found in \cite{CryptoNight}. We will only present its main traits here.

The algorithm relies on random accesses to intrinsically slow memory. It requires around 2 MB of memory, used to store
an ordered list of blocks (of 64 bytes each), each one dependent on the previous one.

The dependency between blocks forces the miner to keep all the blocks in memory, as there is an exponential cost associated to repeated re-calculation of blocks.

The amount of memory needed, 2 MB, fits in modern CPU L3 caches. However, it is far too much for current ASICs. Also, GPU cards
have memory that is slower than CPU cache memory, so they do not bring any advantages in mining tasks.

In sum, mining on common PCs turns out to be perfectly possible with the CryptoNight algorithm.
\\

The hope is that this should yield much less concentration of mining power and as a consequence, a more secure currency.









 


