% this file is called up by main.tex
% content in this file will be fed into the main document

% ---------------------------------------------------------------------------



\chapter{Basic concepts}
\label{chap:basicConcepts}


\section{A few words about notation}

One main objective in this report has been to collect, review, correct and homogenize all existing information
concerning the inner workings of the Monero cryptocurrency.
And at the same time supply all the necessary details to present all this material in a constructive and single-threaded
manner.

An important instrument to achieve this has been to settle for a number of notational conventions that we think
contribute to make this material more readable.
\\

Among others, we have used
\begin{itemize}
\item lower case letters to denote simple values, integers, strings, bit representations, etc
\item upper case letters to denote curve points and complex constructs
\end{itemize}

For items with a special meaning, we have tried to use in as much as possible the same symbols throughout the
whole document. 
For instance,
a curve generator is always denoted by \(G\), its order is \(N\), private/public keys are denoted whenever possible by \(k/K\) 
respectively, etc.
\\

Beyond that, we have aimed at being {\em conceptual} in our presentation of algorithms and schemes.
A reader with a computer science background could consider that we have neglected
 questions like the bit representation of items, or in some cases, how to carry out concrete operations.

However, this does not entail any loss.
A simple object such as an integer or a string can always be represented by a bit string. 
So-called {\em endianness} is rarely relevant, and is mostly a matter of convention for our algorithms.

Elliptic curve points would normally be represented by pairs \((x, y)\) of integers.
However, in the world of cryptography it is common to apply {\em point compression} techniques,
which allow representing a point using only the space of one coordinate.
For our conceptual approach it is quite often accessory whether point compression is used or not.
But most of the times it is implicitly assumed. 
\\

We have also used freely hash functions without specifying any concrete algorithms.
In the case of Monero,  it will typically be a \(\mathit{Keccak}\)\footnote{The Keccak hashing algorithm forms the basis
for the NIST standard {\em SHA-3}} 
variant, but if not explictly mentioned then it's not important to the theory.

These hash functions will be applied to any objects, integers, strings, curve points, or combinations of
these objects. 
These occurrences should be interpreted as hashes of bit representations, or the concatenation of such
representations.
Depending on context, the result of a hash will be numeric, a bit string or even a curve point.
Further details in this respect will be given when needed in this report. 




\section{Elliptic curve cryptography}
\label{EllipticCurveCryptography}

\subsection{What are elliptic curves}


A finite field \(\mathbb{F}_p\) where \(p\) is a prime number, is the field formed by the set \(\{0, 1, 2, ..., p-1\}\). with arithmetic operations \((+,  \cdot)\) calculated \( \pmod p\).


Typically, elliptic curves are defined as the set of points \((x, y)\) satisfying a {\em Weierstraß} equation:
 
\[y^2 = x^3 + a x + b \quad \textrm{where} \quad a, b, x, y \in \mathbb{F}_p \]

However, the cryptocurrency Monero uses a special curve known to offer improved security over other commonly used {\em NIST} curves, as well as excellent performance of cryptographic primitives. The curve used belongs to the category of s.c. {\em Twisted Edwards} curves, which are commonly expressed as: 

\[a x^2 + y^2 = 1 + d x^2 y^2 \quad \textrm{where} \quad a, d, x, y \in \mathbb{F}_p \]


In what follows we will prefer this second form. The advantage it offers over the previously mentioned Weierstraß form is that basic cryptographic primitives require less arithmetic operations. This results in faster cryptographic algorithms, see Bernstein et al. in \cite{Bernstein2007} for details.
\\

Let \(P_1 = (x_1, y_1)\) and \(P_2 = (x_2, y_2)\) be 2 points belonging to an elliptic curve. We can proceed to define the addition operation \(P_1 + P_2 = P_3\) in the following manner

\begin{align*}
x_3 & =  \frac{x_1 y_2 + y_1 x_ 2}{1 + d x_1 x_2 y_1 y_2}  \pmod p \\
y_3 & =  \frac{y_1 y_2 - a x_1 x_2}{1 - d x_1 x_2 y_1 y_2} \pmod p 
\end{align*}

These formulas for addition will also apply for point doubling, that is, when \(P_1 = P_2\).
\\

It turns out that elliptic curves have {\em abelian group} structure under the addition operation described\footnote{
A concise definition of this notion can be found under \url{https://brilliant.org/wiki/abelian-group/}}.


The order \(N\) of an elliptic curve \(EC\) can be loosely defined as the number of points in the curve. Clearly, by the definition of addition above it follows that an elliptic curve over a finite field will also be finite.  


A generator \(G\) of \(EC\) is a point in the curve such that for every other point \(P\) in \(EC\) there exists \(n\) satisfying that \(P = n G\).

Calculating the scalar product \(n P\) can be done without difficulties. However, given two arbitrary points \(P_1\) and \(P_2\), finding \(n\) such that \(P_1 = n P_2\) is known to be computationally hard. By analogy to modular arithmetic, this problem is often called the {\em discrete logarithm problem} (DLP). In other words, scalar multiplication can be seen as a {\em one-way function}, which paves the way
for using elliptic curves for cryptography. 


\subsection{Public key cryptography with elliptic curves}
\label{ec:keys}
Public key cryptography algorithms can be devised in an analogous way as with modular arithmetic. 

Let \(k\) be a randomly selected number satisfying \(1 < k < N\), and call it {\em private key}. 
Calculate the corresponding {\em public key} \(K = k G\). 

Due to the hardness of the {\em discrete logarithm problem} (DLP) we can not easily deduce the value \(k\) from \(K\) alone.
This property allows us to use the values in \( (k, K) \) in common public key cryptography algorithms.


\subsection{Diffie-Hellman key exchange with elliptic curves}
A basic {\em Diffie-Hellman} exchange of a shared secret between {\em Alice} and {\em Bob}  could take place in the following manner:

\begin{enumerate}
	
	\item Alice and Bob generate their own private/public keys \((k_A, K_A) \quad \textrm{and} \quad (k_B, K_B)\). Both publish or exchange their public keys, and keep the private keys for themselves.
	
	\item Clearly, it holds that \[S = k_A K_B = k_A k_B G = k_B k_A G = k_B K_A\]
	 Therefore, Alice could calculate \(S = k_A K_B\) and Bob \(S = k_B K_A\),  and use this single value as a shared secret.
	
\end{enumerate}   

An external observer would not be able to calculate easily the shared secret due to the hardness of the DLP.


\subsection{DSA signatures with elliptic curves (ECDSA)}

Typically, a cryptographic signature is performed on a cryptographic hash of a message rather than the message itself.
However, in this whole report we will loosely use the term {\em message} indistinctly to refer to the message properly speaking and/or its hash value.

\subsubsection*{Signature}

Assume that Alice has the private/public key pair \((k, K)\). To sign univocally an arbitrary message \(\mathfrak{m}\), she could execute the following steps \cite{Hankerson:2003:GEC:940321}:

\begin{enumerate}
	
	\item Calculate a hash of the message using a cryptographically secure hash function, \(h = \mathcal{H}(\mathfrak{m})\)
	
	\item Generate a random integer \(r\) such that \(1 < r < N\) and compute \(P = (x, y) = r G\). 

       If \(x = 0\) generate another random integer.
	
	\item Calculate \(s = r^{-1} (h + x k) \pmod N\). If \(s = 0\) then go to previous step and repeat
	
	\item The signature is \((x, s)\)
\end{enumerate}


\subsubsection*{Verification}

Any third party can verify the signature by calculating
\begin{align*}
  u_1 & = s^{-1} h \\
  u_2 & = s^{-1} x \\
  Q   & = u_1 G + u_2 K
\end{align*}

The signature will be valid if and only if the first coordinate of \(Q = (x_Q, y_Q)\) satisfies that  
\[x_Q = x \pmod p\]. 

\subsubsection*{Correctness}

The correctness of the scheme can be derived from the fact that
\begin{align*}
Q &= u_1 G + u_2 K \\
  &= s^{-1} h G + s^{ -1} x k G \\
  &= s^{-1} (h + x k) G \\
\end{align*}

Since \(s = r^{-1} (h + x k) \), it follows that \(r = s^{-1}(h + x k)\), whereby it becomes proved that \[Q = r G\].

\section{Curve Ed25519}

Monero uses a particular Twisted Edwards elliptic curve for cryptographic operations, {\em Ed25519}, {\em birational equivalent}\footnote{Without giving further details, birational equivalence can be thought of as an isomorphism that can be expressed using rational terms} 
of the Montogomery curve {\em Curve25519}.

Both Curve25519 and Ed25519 were released by Bernstein {\em et al.} \cite{Bernstein2008, Bernstein2012, cryptoeprint:2007:286}.

The curve is defined over the prime field \(\mathbb{F}_{2^{255} - 19} \) by means of the following equation:

\[ -x^2 + y^2 = 1 - \frac{121665}{121666} x^2 y^2 \]

This curve addresses many concerns raised by the cryptography community. It is well known that {\em NIST}\footnote{National Institute of Standards and Technology, \url{https://www.nist.gov/}} 
standard algorithms have issues. In recent times it has become clear that the random number generation algorithm {\em PNRG} is flawed and contains a potential backdoor\cite{hales2014nsa}. Seen from a broader perspective, curves endorsed by the NIST, are also indirectly endorsed by the NSA, something that the cryptography community sees with suspicion. It
is well known that NSA has at times used its power over the NIST to weaken cryptographic algorithms\cite{NSA-NIST}.
\\

Curve Ed25519 is not subject to any patents (see \cite{ECC-patents} for a discussion on this subject), and the team behind it has
developed and adapted basic cryptographic algorithms with efficiency in mind \cite{cryptoeprint:2007:286}.
More importantly, it is currently thought to be secure.

Twisted Edwards curves have order expressable as \(2^c q\), where \(q\) is a prime number and \(c\) a positive integer. In the case of curve Ed25519, its order is a 76 decimal digit number:
\[2^3 \cdot 7237005577332262213973186563042994240857116359379907606001950938285454250989\]


\subsection{Binary representation}
Elements of \(\mathbb{F}_{2^{255} - 19} \) are 256-bit integers. In other words, they can be
represented using 32 bytes.

Consequently, any point in Ed25519 could be represented using 64 bytes.
Applying {\em Point compression} techniques, described here below, however, it is possible to reduce this amount by half, to 32 bytes.

\subsection{Point compression}

The Ed25519 curve has the property that its points can be easily compressed,
so that representing a point will consume only the space of one coordinate.
We will not delve into the mathematics necessary to justify this, but we can
give a brief insight into how it works.

Assume that we want to compress a point \((x, y)\).
We will employ a little-endian representation of integers.

\begin{description}
	
	
	\item[Encoding] The most significant bit of the \(y\) coordinate will always be \(0\).
	We set this bit to 0 if \(x\) is even, and \(1\) if it is odd.
	The resulting value \(y\) will represent the curve point.
	
	\item[Decoding] Retrieve and clear the most significant bit of the stored value and that will be \(y\).\\
	 Let \(u = y^2-1\) and \(v = d y^2  + 1\). \\
	 Compute \(x = u v^3 (u v^7)^{(p-5)/8}\)
	 \begin{enumerate}
	 \item If \(v x^2 = u \pmod p\) then keep \(x\)
	 \item Else set \(x = x 2^{(p-1)/4} \pmod p\)
	 \item Use the parity bit \(b\) retrieved in the first step, if \(b \ne x \pmod p \) then return \(p - x\),
	 otherwise return \(x\)
	 \end{enumerate}	
	
\end{description}


\subsection{EdDSA signature algorithm}

Bernstein and his team have developed a number of basic algorithms based on curve Ed25519.
For illustration purposes we will describe here a highly  optimized and secure alternative to the ECDSA signature scheme
which according to the authors allows producing over 100 000 signatures per second using a commodity Intel Xeon processor \cite{Bernstein2012}.
The algorithm can also be found described in Internet RFC8032 \cite{rfc8032}.

Among other things, instead of generating random integers every time, it uses a hash value derived from the private key of the signer, 
and the message itself. This circumvents security flaws related to the implementation of random number generators.
Also, another goal of the algorithm is to avoid accessing secret or unpredictable memory locations to prevent so-called {\em cache timing attacks},

We provide here an outline of the steps performed by the algorithm for illustration purposes only.
A complete description and sample implementation in the Python language can be found in \cite{rfc8032}. 

\subsubsection{Signature}

\begin{enumerate}
	
	\item Let \(h_k\) be a hash \(\mathcal{H}(k)\)of the signer's private key \(k\). 
	Compute \(r\) as a hash \(r = \mathcal{H}(h_k,  \mathfrak{m})\)  of the hashed private key and message.
	
	\item Calculate \(R = r G\) and \(s = (r + \mathcal{H}(R, K,  \mathfrak{m}) \cdot k)) \)
	
	\item the signature is the pair \((R, s)\)
	
	
\end{enumerate}



\subsubsection{Verification}
Verification is performed as follows

\begin{enumerate}
	
	\item Compute \(h = \mathcal{H}(R,  K,  \mathfrak{m}) \)
	
	\item If the equality \((2^c s) G = 2^c R + 2^c \mathcal{H}(R, K,  \mathfrak{m}) K \) holds then the signature is valid
	
\end{enumerate}



\subsubsection{Correctness}

Why the signature verification works can be derived from the following equality:

\[ 2^c s G = 2^c ( (r + \mathcal{H}(R, K,  \mathfrak{m}) \cdot k) \cdot G = 2^c R + 2^c \mathcal{H}(R, K,  \mathfrak{m}) \cdot K )\]



\subsubsection{Binary representation}

By default, an EdDSA signature would need \(64 + 32 \) bytes to be represented.
However, RFC8032 assumes that point \(R\) is compressed, whereby space requirements become reduced to \(32 + 32\) bytes only.

